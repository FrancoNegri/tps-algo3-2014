\documentclass[a4paper,11pt]{report}

\addtolength{\textwidth}{5.2cm}
\addtolength{\voffset}{-3cm}
\addtolength{\hoffset}{-2.5cm}
\addtolength{\textheight}{5cm}
\addtolength{\headheight}{14.0pt}


% \usepackage[utf8]{inputenc}
\usepackage{amsfonts}
\usepackage{amsmath}
\usepackage{amssymb}
\usepackage{amsthm} % para los teoremas
\usepackage[T1]{fontenc}
\usepackage[latin1]{inputenc}
\usepackage[spanish]{babel}
\usepackage{caratula}
\usepackage{framed}
\usepackage{caption}
% \usepackage{newalgo3}
%\usepackage{otrosComandos}
\usepackage{fancyhdr}
\usepackage{lastpage}
%\usepackage{hyperref}
\usepackage[pdfborder={0 0 0}]{hyperref}
\usepackage{url}
\usepackage{graphicx}
\usepackage{tikz} %viene del paquete pgf (graficar adentro de latex)
\usepackage{listings} % es para resaltado de codigo fuente y pseudocodigos
\usepackage[noend]{algorithmic} % es para pseudocodigos
% \usepackage[ruled,lined,algonl]{algorithm2e}
\usepackage{algorithm} % es para pseudocodigos
\usepackage{enumerate} % para que los items salgan lindos
\usepackage{titlesec}
\usepackage{colortbl}
\usepackage{xcolor}
\usepackage{graphicx}
\sloppy
\def\infinity{\rotatebox{90}{8}}
%\usepackage[table]{xcolor}
% Defino el estilo para el paquete amsthm --------------------
\theoremstyle{plain}% default
\newtheorem{thm}{Teorema}[section]
\newtheorem{lem}[thm]{Lema}
\newtheorem{prop}[thm]{Proposici\'on}
\newtheorem*{cor}{Corolario}
\newtheorem*{KL}{Klein's Lemma}

\theoremstyle{definition}
\newtheorem{defn}{Definici\'on}[section]
\newtheorem{conj}{Conjectura}[section]
\newtheorem{exmp}{Ejemlo}[section]
%-------------------------------------------------
\usepackage{eqparbox}
\newcommand{\asd}[1]{\hfill\eqparbox{COMMENT}{#1}\\}

\newcommand{\tOde}[1]{O(#1)}
\newcommand{\Ode}[1]{O(#1)}

%\usetikzlibrary{chains} % para ubicar nodos en una cadena (lineal, circular o cualquier otra)
%\usetikzlibrary{arrows,%
 %               petri,%
  %              topaths}%

% ----Cosas que usa listings ---------------
\usepackage{color}
\definecolor{gray97}{gray}{.97}
\definecolor{gray75}{gray}{.75}
\definecolor{gray45}{gray}{.45}

\lstset{ frame=Ltb,
     framerule=0pt,
     aboveskip=0.5cm,
     framextopmargin=3pt,
     framexbottommargin=3pt,
     framexleftmargin=0.4cm,
     framesep=0pt,
     rulesep=.4pt,
     backgroundcolor=\color{gray97},
     rulesepcolor=\color{black},
     %
     stringstyle=\ttfamily,
     showstringspaces = false,
     basicstyle=\small\ttfamily,
     commentstyle=\color{gray45},
     keywordstyle=\bfseries,
     %
     numbers=left,
     numbersep=15pt,
     numberstyle=\tiny,
     numberfirstline = false,
     breaklines=true,
   }

% minimizar fragmentado de listados
\lstnewenvironment{listing}[1][]
   {\lstset{#1}\pagebreak[0]}{\pagebreak[0]}

% Para formatear la consola
\lstdefinestyle{consola}
   {basicstyle=\small\bf\ttfamily,
    backgroundcolor=\color{gray97},
   }


%------Fin cosas que usa listings -------------------

% valores razonables para las figuras (para que no aparezcan figuras solas
\renewcommand{\topfraction}{0.85}
\renewcommand{\textfraction}{0.1}
\renewcommand{\floatpagefraction}{0.75} % este valor no debe superar topfraction

\date{7/9/12}


%Redefino algunos nombres (como uso babel, lo tengo que hacer asi):
% Donde dice \chapter quiero que aparezca Problema en lugar de Capitulo

%Donde dice Indice general, quiero que aparezca Contenidos
\addto\captionsspanish{% esto lo necesito porque uso babel
  \renewcommand{\contentsname}%
    {Contenidos}%
}

%Donde dice Indice general, quiero que aparezca Contenidos
\addto\captionsspanish{% esto lo necesito porque uso babel
  \renewcommand{\appendixname}%
    {Anexo}%
}

%Donde dice Cuadro, quiero que aparezca Tabla
\addto\captionsspanish{% esto lo necesito porque uso babel
  \renewcommand{\tablename}%
    {Tabla}%
}
% Paquetes para dibujar los ejes de coordenadas del ejercicio LaCajaEnElPlano

% \usepackage{multido}
% \usepackage{pstricks-add}
% \usepackage{pst-node}
% \usepackage{pst-plot}
% \usepackage{pstcol}
% \usepackage{pst-all}




%Algunas definiciones
%marca el numero de footnote actual: \marcar{nombre} nombre es un counter nuevo
\def\marcar#1{\newcounter{#1}\setcounter{#1}{\value{footnote}}}
%para hacer referencia a una footnote marcada: \nota{contador}
\newcounter{auxiliar} %lo uso para guardar footnote
\def\nota#1{\setcounter{auxiliar}{\value{footnote}}\footnotemark[\value{#1}]\setcounter{footnote}{\value{auxiliar}}}

\begin{document}
    \materia{Algoritmos y Estructuras de Datos III}

    \titulo{Trabajo Práctico 1}

    \subtitulo{Segundo Cuatrimestre 2014}

    \grupo{}
    \integrante{Ricardo Colombo}{156/08}{ricardogcolombo@gmail.com.com}
    \integrante{Federico Suarez}{610/11}{elgeniofederico@gmail.com}
    \integrante{Juan Carlos  Giudici}{827/06}{elchudi@gmail.com}
    \integrante{Franco Negri}{000/00}{franconegri2004@gmail.com}


    \maketitle

%\markright{}

\tableofcontents
%\pagebreak
\pagestyle{fancy}
\rfoot{}
\cfoot{\thepage ~ de \pageref{LastPage}}
\lfoot{}
\rhead{}
\lhead{\nouppercase\leftmark}
\chead{}



\chapter{Puentes sobre lava caliente}  
\subsection{Introducci\'on} 

En este ejercicio se nos pide encontrar un algoritmo que encuentre una combinacion de vuelos entre la ciudad $A$ y la ciudad $B$ tal que encuentre la manera de llegar lo antes posible a destino.

La entrada del algoritmo ser\'a:

\begin{itemize}
\item Un string \textbf{A} $\rightarrow$ Representar\'a la ciudad de partida.
\item Un string \textbf{B} $\rightarrow$ Representar\'a la ciudad de de llegada.
\item Un entero \textbf{n} $\rightarrow$ Representar\'a el numero total de vuelos entre todas las ciudades.
\item \textbf{n} filas donde, para cada fila se tiene:
\begin{itemize}
\item Un string \textbf{ori} $\rightarrow$ Representar\'a la ciudad de partida.
\item Un string \textbf{des} $\rightarrow$ Representar\'a la ciudad de de llegada.
\item Un entero \textbf{ini} $\rightarrow$ Representar\'a el numero la hora de despegue de la ciudad $ori$
\item Un entero \textbf{fin} $\rightarrow$ Representar\'a el numero la hora de arribo a la ciudad $des$
\end{itemize}
\end{itemize}

A esto nuestro algoritmo deve devolver:
\begin{itemize}
\item Un entero \textbf{fin} $\rightarrow$ Representar\'a el horario de llegada a la ciudad $B$.
\item Un entero \textbf{k} $\rightarrow$ la cantidad de vuelos del itinerario.
\item $k$ enteros \textbf{v_1, v_2 ..., v_k} $\rightarrow$ los vuelos tomados
\end{itemize}

\subsection{Ejemplos y Soluciones}
Se procede a generar una posible instancia del problema para ilustrar lo que se espera del algoritmo.

Supongamos que queremos ir de la ciudad de Buenos Aires a la isla de Saba (Dato curioso: en la isla de Saba se encuentra el aeropuerto mas chico del mundo (400 m)).

Lamentablemente no existen vuelos directos, por lo que tendrémos que hacer escala para poder llegar ahí. A continuaíón se muestran los posibles vuelos que podríamos tomar para llegar a destino.

\begin{itemize}
\item Buenos Aires - Seúl: 10 - 20
\item Buenos Aires - La isla de los pitufos: 17 - 24
\item Seúl - San Fransisco: 22 - 40
\item La isla de los pitufos - isla de Saba: 28 - 30
\item San Fransisco - isla de Saba: 40 - 43
\item Buenos Aires - San Fransisco: 13 - 20
\end{itemize}

Luego, podemos observar que existen muchas maneras factibles de llegar desde Buenos Aires al objetivo. Una manera es tomar la ruta de Buenos aires - Seul, de allí ir a San Fransisco y finalmente a la isla de Saba, pero rapidamente vemos que si bien este viaje es factible, es al menos tan bueno como tomar el vuelo de Buenos aires a san fransisco y de allí continuar a Saba. De esta observación se desprende un dato interesante que utilizaremos en nuestro algoritmo, si se llegar de la manera mas barata a un grupo de ciudades $v_1$, $v_2$, ... $v_s$ y tengo $w_1$, $w_2$,...$w_i$ ciudades a las que puedo llegar desde el primer grupo. Si ahora tomo la ciudad $w_g$ tal que la hora de arribo a esa ciudad desde una en $v$ es la menor posible, sé que no es posible llegar de una manera mas barata a esa ciudad tambien.

Rapidamente esta idea nos remite al algoritmo de dijstra, el cual comparte una gran similitud en la idea de ir explorando todos los caminos más cortos que parten del vértice origen y que llevan a todos los demás vértices hasta llegar al destino, por lo que mas adelante intentaremos valernos de esta idea.

Luego de esta gran revelación algoritmica, continuamos con el problema que nos atañe. Es facil resolver a mano esta instancia del problema y descubrir que la combinación de vuelos optima es:
\begin{itemize}
\item Buenos Aires - La isla de los pitufos: 17 - 24
\item La isla de los pitufos - isla de Saba: 28 - 30
\end{itemize}
Llegando a destino a la hora $30$.

\subsection{Desarrollo}
Como ya hemos adelantado en el apartado anterior, la idea subyacente en este problema es, utilizando un algoritmo de dijstra levemente modificado, en cada paso, calcular la manera mas barata de llegar desde un conjuto de ciudades ya visitadas, a uno quue todavía no hemos visitado.

==COMPLETAR==

\subsection{Complejidad}

Dado que el algoritmo consiste en iterar por dos loops anidados, cuyo tamaño es $n$ (osea, la cantidad de vuelos), la compelgidad del algoritmo será $O(n^2)$.

Ademas, este algoritmo para cada paso del loop mas externo, itera forzosamente por todo el arreglo para determinar el minimo, por lo que sabemos que, para el caso en que existe una solución $\omega (n^2)$.

\subsection{Experimentacion}

Dado que ya dijimos que no existen, 'peores casos', porque nuestro algoritmo esta acotado por ambos lados por $n^2$, realizamos un testeo random para comprobarlo:

\begin{figure}[h!]
  \begin{center}
	\includegraphics[scale=0.5]{Ej1/testingnvst.png}
	%\caption{Descripcion de la figura}
	\label{nombreparareferenciar}
  \end{center}
\end{figure}

Aqui puede verse claramente que nuestras hipotesis eran correctas.

\chapter{Horizontes lejanos}  
\subsection{Introduccion} 
Se esta diseñando un software de arquitectura, para el cual es necesario que dado un conjunto de edificios representados como rectangulos apoyados sobre una base en comun, se devuelva el perfil definido en el horizonte.\\
Estos edificios vienen representados por tuplas de tres elementos que representan donde comienza el edificio, su altura y donde termina, de las cuales tenemos que ir tomando en cada momento donde comienza un edificio la altura maxima alcanzada en ese punto.

 \subsection{Ejemplos y Soluciones}
 Consideremos el siguiente ejemplo del problema:\\
 $ [<3,2,5>;<1,4,2>;<4,1,6>;<6,8,10>]$ \\
  Cada una de estas tuplas de tres elementos se indica donde comienza el edificio en la primera coordenada, su altura en la segunda y donde termina en la tercera coordenada.\\
  
  Lo primero que hacemos es ordenar estas tuplas en orden creciente por lo que representa donde comienza el edificio (llamemosla pared izquierda), quedandonos de la siguiente manera:\\ 
$[<1,4,2>;<3,2,5>;<4,1,6>;<6,8,10>$] \\

Por otro lado ordenamos los edificios por la coordenada donde terminan (llamemosla pared derecha).\\
$[<1,4,2>;<3,2,5>;<4,1,6>;<6,8,10>$] \\

\subsection{Desarrollo}
Como mencionamos anteriormente, tenemos como datos de entrada la posici\'on donde comienza y termina cada edificio y adem\'as su altura, con lo cual representaremos a los edificios con tuplas de 3 elementos (posici\'on de inicio o pared izquierda, altura, y posici\'on donde termina o pared derecha). Primero organizaremos a los edificios en dos arreglos, donde cada arreglo contendr\'a el total de edificios, uno con un orden ascendente seg\'un pared izquierda y el otro tambi\'en con un orden ascendente pero seg\'un pared derecha. Adem\'as tendremos un conjunto en el que iremos agregando los edificios que ``comience'' y quitando los edificios que ``terminen'', es decir, aqu\'i� estar\'an los edificios ``activos'' (que ``empezaron'' y no ``terminaron'') en cada momento.
La idea del algoritmo es ir recorriendo las posiciones (del eje x) en las que haya una o m\'as paredes. En cada punto lo que haremos es agregar a mi conjunto de activos los edificios que en ese punto tengan su pared izquierda, o sea que est\'an "comenzando", y quitar a los edificios que all\'i tengan su pared derecha, o sea que est\'an ``terminando''. Una vez completada esta labor, buscaremos el edificio activo que tenga la altura m\'axima y nos lo guardaremos, llam\'emoslo Max. Se puede ver claramente que el borde superior de la silueta en un punto dado va a estar determinado por el edificio m\'as alto que haya en ese punto, es decir, el edificio activo m\'as alto. Como en cada paso podemos conseguir el edificio activo m\'as alto, proseguiremos as\'i� hasta el \'ultimo punto y habremos arm\'andonos la silueta.

\subsection{Demostraci\'on}
Demostraremos por inducci�n que en cada paso de este algoritmo obtenemos la altura del edificio m�s alto en ese punto.

P(i) = "Nuestro conjunto de edificios activos contiene todos los edificios que empezaron entre el punto de inicio y el punto i inclusive y que a�n no han terminado, es decir, terminan en un punto estrictamente mayor que i"

Caso base, P(1):
En este caso nos encontramos con nuestro primer conjunto de paredes, que puede tener una o m�s paredes. Este s�lo puede tener paredes izquierda ya que es donde comienzan nuestros primeros edificios y ning�n edificio ha empezado antes como para que aparezca una pared derecha indicando su finalizaci�n. Por lo tanto s�lo tenemos un conjunto de edificios que comienzan en este punto, los cuales agregaremos a nuestro conjunto de edificios activos. Es claro ver que el conjunto de edificios activos ahora tiene todos los edificios que comenzaron y no han terminado hasta el primer punto inclusive ya que ning�n edificio termina aqu� como previamente hemos dicho y entonces todos estos empezaron y no terminaron.

Paso inductivo, P(n) $\rightarrow$ P(n+1):
Nuestra hip�tesis inductiva nos dice que nuestro conjunto de edificios activos contiene todos los edificios que empezaron entre el punto de inicio y el punto n inclusive, y que a�n no han terminado. Ahora veamos qu� ocurre en n+1. En este punto podemos encontrar un conjunto que contiene tanto paredes izquierda como derecha, con lo cual separaremos a este conjunto de paredes en dos subconjuntos. Por un lado el conjunto de paredes izquierda, llam�mosle I, y por otro el de paredes derecha, llam�mosle D. Recorreremos primero el conjunto I agregando cada edificio de este conjunto al conjunto de edificios activos, es decir, agregaremos todos los edificios que comienzan en este punto.
Acto seguido, recorreremos el conjunto D quitando cada edificio que aparezca en este conjunto de nuestro conjunto de edificios activos ya que todos estos edificios est�n terminando y por lo tanto ya no deben pertenecer a nuestro conjunto. Veamos ahora que efectivamente se est� cumpliendo P(n+1). Los edificios agregados en este paso no pueden terminar aqu� mismo, ya que eso implicar�a que est�n empezando y terminando en la misma posici�n, lo cual no es v�lido. Por lo tanto estar�an terminando en un punto estrictamente mayor a n+1. 
Los edificios que terminaban en el paso n+1 fueron ya removidos del conjunto. Ahora miremos qu� pasa con los dem�s edificios que ya estaban en nuestro conjunto de edificios activos y no fueron eliminados. Por hip�tesis inductiva, estos edificios no tienen su pared derecha entre el punto de inicio y el punto n, es decir, terminan en un punto estrictamente mayor que n. Pero como no fueron borrados de nuestro conjunto, quiere decir que no tienen su pared derecha en n+1, o sea que terminan en un punto estrictamente mayor a n+1. Entonces vale P(n+1).


Teniendo nuestro conjunto de edificios activos, en cada punto podemos buscar el edificio de mayor altura y registrar los cambios de altura cada vez que empieza y/o termina un edificio. Luego, es trivial armarnos la silueta ya que, con las fluctuaciones de las alturas m�ximas a lo largo de nuestra ciudad, ya tenemos su borde superior en cada tramo.


\subsection{Complejidad}

definimos edificio = < izq :natural x alto : natural x der : natural >
definimos edificioenCero al que tiene todos sus elementos en cero.
\begin{algorithm}
\begin{algorithmic}[1]\parskip=1mm
 \caption{LaSilueta( ciudad: arreglo(edificios) , cantidadEdificios : natural)}
	
	\STATE{arregloXIzq \leftarrow ordenarXIzquierda(ciudad)}\\
	\STATE{arregloXDer \leftarrow ordenarXDerecha(ciudad)}\\
	\STATE{EdificiosActivos \leftarrow Multiconjunto(edificio)}\\
	\\
	\STATE{posIzquierdo , posDerecho \leftarrow 0}\\
	\STATE{max \leftarrow edificioenCero \\
	\STATE{mientras posIzquierdo < cantidadEdificios \&\& posDerecho < cantidadEdificios }\\
	\STATE{ \quad SI arregloXIzq.indice(posIquierdo).izq \leq arregloXDer.indice(posDerecho).der}\\
			\STATE{\quad\quad auxiliar \leftarrow arregloXIzq.indice(posIzquierdo)}\\
			\STATE{\quad\quad mientras auxiliar.izq = arregloXIzq.indice(posIquierdo).izq \&\& posIzquierdo != cantidadEdificios}\\
			\STATE{\quad\quad\quad agregar(arregloXIzq.indice(posIquierdo), EdificiosActivos )}\\
			\STATE{\quad\quad\quad SI arregloXIzq.indice(posIquierdo).alto > max.alto}\\
			\STATE{\quad\quad\quad\quad max \leftarrow arregloXIzq.indice(posIquierdo) }\\
		\STATE{\quad SI NO }\\
			\STATE{\quad\quad auxiliar \leftarrow arregloXDer.indice(posDerecha)}\\
			\STATE{\quad\quad mientras auxiliar.der = arregloXDer.indice(posDerecha).der && posDerecha != cantidadEdificios}\\
			\STATE{\quad\quad\quad SI  arregloXDer.indice(posDerecha).id = max.id}\\
			\STATE{\quad\quad\quad\quad dameMaximoSiguiente( arregloXDer.indice(posDerecha), EdificiosActivos)}\\
			\STATE{{\quad\quad\quad\quad sacar( arregloXDer.indice(posDerecha), EdificiosActivos)}
\\
\end{algorithmic}
Los algoritmos ordenarXizquierda y ordenarXDerecha es el conocido algoritmo mergeSort sacado del libro de brassard, la complejidad del mismo es \Ode{N*Log(N)} siendo N la cantidad de elementos en el arreglo, �sea todos los edificios, estos algoritmos ordenan los arreglos de tuplas, uno por la coordenada izq y el otro por la coordenada derecha respectivamente.\\
El multiconjunto EdificiosActivos esta representado con la estructura Multiset de la librer�a STL de c++, la relaci\'on de orden sobre los elementos que se defini\'o es por la coordenada alt y en el caso que la altura sea igual por la componente der.\\
Para las operaciones para agregar, sacar elementos la complejidad es \Ode{log(n)} siendo n la cantidad de elementos en el multiconjunto y para obtener el proximo maximo lo que se realiza es obtener la referencia al elemento que 
representa el m\'aximo, esto lleva \Ode{Log(n)} y luego se obtiene el posterior, en caso de no ser posible obtenemos el anterior, en el peor caso se tienen todos los edificios y la complejidad seria \Ode{log(N)}.\\
El ciclo de las lineas 9-12 se realiza para cada punto x donde haya paredes izquierdas, con lo cual la complejidad total sera la suma de todas las paredes izquierda, siendo esta N la cantidad de edificios. De la misma manera el ciclo de las lineas 15 a 18 se ejecuta para todas las paredes derechas de los edificios con lo cual el total de las iteraciones sera la suma de las paredes derechas �sea N. Como el ciclo principal se ejecuta mientras la cantidad de paredes recorridas izquierdas sea menor a la cantidad de edificios y la cantidad de paredes derechas sea menor a la cantidad de edificios y las iteraciones internas modifican las paredes izquierdas y derechas recorridas, el total de las iteraciones es 2*N, siendo N la cantidad total de edificios, con lo cual como la complejidad de agregar , sacar y obtener el m\'aximo es \Ode{log(N)} en el peor caso, el total es \Ode{2*N*Log(N)} $\subset$ \Ode{N*Log(N)} siendo esta la complejidad total del algoritmo y esta estrictamente menor a \Ode{N^2} como se solicitaba en el enunciado.

\end{algorithm}	
\newpage
\subsection{Experimentaci\'on}
Se realizaron experimentaciones sobre varios tipos de caso para observar el comportamiento del algoritmo, en todos los casos se generaron 1000 entradas, a las gr�ficas resultantes se las comparo con la funci�n n*log(n)*1000, siendo n la cantidad de edificios de entrada.\\
Para la primer ejecuci�n se realizo un an\'alisis sobre el algoritmo en edificios donde la distancia entre sus paredes izquierda y derecha era corta, estas se encontraban entre las posiciones 1 y 30, pero sus alturas variaban en el orden entre 1 y 10000, dando el siguiente gr�fico como resultado.\\

\includegraphics[scale=0.7]{Ej2/VariacionAlturas.jpg}\\

El segundo gr\'afico corresponde a instancias generadas de manera random donde la altura se encontraba entre 1 y 40, pero las distancias entre la primer y la segunda pared de cada edificio varia entre los 1 y 2000 , elegido de manera random la posici�n de ambos lados pero siempre que sea entrada valida, \'osea el valor de la pared izquierda es mayor estricta a la de la pared derecha.

\includegraphics[scale=0.7]{Ej2/VariacionDistancias.jpg}


Luego para el tercer gr\'afico consideramos que era importante ver como se comportaba el multiconjunto cuando se agregaban todos los edificios de la entrada, de esta manera se generaron entradas donde los edificios eran todos iguales, as\'i se ingresaban todos al multiconjunto y ve\'iamos como se comportaba el algoritmo, obteniendo este gr\'afico como resultado donde se ve que su complejidad es del orden logar\'itmico en la cantidad de edificios de entrada por la cantidad edificios.

\includegraphics[scale=0.7]{Ej2/iguales.jpg}\\
Por ultimo realizamos un gr�fico con entradas random donde las mismas variaban combinando todas las anteriores, obteniendo de nuevo que el orden era logar�tmico al compararlo con la funci�n.

\includegraphics[scale=0.7]{Ej2/Random.jpg}\\

 Concluimos por lo tanto que el algoritmo respeta los ordenes de complejidad requeridos.





\chapter{Biohazard} 
\subsection{Introducci\'on} 

En este ejercicio se propone solucionar el problema de dado una red de existente de computadoras, con a lo sumo un enlace entre cada par de ellas y con un costo asociado al mismo, seleccionar algunas de estas para formar un backbone con topología de red tipo anillo, la cual tiene que tener como característica que conecte a todas las computadoras originales y que minimice el costo de ancho de banda de la red.

Se pide un algoritmo que genere este backbone, con un costo temporal estrictamente menor que O($n^3$), este algoritmo debe detectar casos en los que no hay solucion.

\subsection{Desarrollo}

\subsubsection{Modelado}

Dada una red, la misma se puede modelar con un grafo de la siguiente manera:

\begin{enumerate}
	\item Cada computadora se representa con un nodo.
	\item Los enlaces entre cada par de computadoras se son los ejes en mi grafo, con el costo de ancho de banda como peso del mismo.
\end{enumerate}

Transformamos el problema de ser uno de redes, a uno de grafos, donde encontrar un backbone con topologia de red tipo anillo que tenga costo mínimo y conecte a todas las computadoras, se transforma en encontrar subgrafo conexo cuyo costo sea mínimo y que contenga un único circuito simple, el cual se corresponde al backbone.

\subsubsection{Solucio'n, Correctitud y Complejidad}

Suponiendo que el grafo resultante es conexo, ya que si no lo fuese no habria soluci'on, vamos a utilizar el algoritmo de Prim, el cual dado un grafo conexo con pesos asociados a sus ejes construye un Árbol Generador Mínimo, es decir un subgrafo conexo cuya la suma de los pesos de sus ejes es mínima. Para completar el circuito, seleccionamos el eje con costo mínimo de los no elegidos por el algoritmo de Prim, el cual nos va a generar un único circuito simple.

Este último paso se justifica por lo visto en las clases teóricas donde demostramos que dado un Árbol, si agregamos un eje entre cualquier par de nodos, se forma un único circuito simple.

El algoritmo es el siguiente:

\begin{algorithm}
\begin{algorithmic}[1]\parskip=1mm
\caption{EncontrarBackBone( G(E,V) ) }
	\STATE{si no Es_conexo_o_no_tiene_ejes_suficientes_para_construir_un_circuito(G)}	
	\STATE ~~~{ devolver no}
	\STATE{agm, ejes_no_seleccionados $\leftarrow$ Prim(G)}
	\STATE{eje_minimo $\leftarrow$ Encontrar_Eje_Minimo(ejes_no_seleccionados)}
	\STATE{circuito $\leftarrow$ Construir_Circuito(agm, eje_minimo)}
	\STATE{Mostrar circuito}
 \end{algorithmic}
\end{algorithm}

\begin{itemize}
	\item Nuestra implementación del algoritmo de Prim, ademas del árbol generador mínimo genera una lista de ejes no seleccionados, la cual vamos a usar para encontrar el eje mínimo y completar el circuito.
	\item \verb+Es_conexo_o_no_tiene_ejes_suficientes_para_construir_un_circuito(G)+ recorre el grafo mediante DFS, llevando la cuenta de los nodos visitados para decidir si es conexo una vez finalizado y para decidir si es posible armar un circuito verifica que \verb+m >= n+.
	\item \verb+Encontrar_Eje_Minimo(ejes_no_seleccionados)+ busca linealmente en la cantidad de ejes (a lo sumo $m = n^2$) el eje con costo mínimo.
	\item \verb+Construir_Circuito(agm, eje_minimo)+ Toma como punto principio y final los nodos del eje_minimo y mediante DFS construye el circuito.
\end{itemize}

\subsubsection{Complejidad}

La complejidad del algoritmo \verb+EncontrarBackBone+  es la siguiente: 

\begin{itemize}
	\item \verb+Es_conexo_o_no_tiene_ejese_suficientes_para_construir_un_circuito(G)+ tiene un costo de O($n^2$), esta implementado mediante una variacion del algoritmo de DFS, el cual va marcando los nodos visitados.
	\item La implementación de \verb+Prim(G)+ que utilizamos tiene un costo O($n^2$), dado que utilizamos una matriz de adyacencias.
	\item \verb+Encontrar_Eje_Minimo(ejes_no_seleccionados)+ tiene un costo de O($n^2$)
	\item \verb+Construir_Circuito(agm, eje_minimo)+ tiene un costo de O($n$), dado que el agm tiene $n-1$ aristas.
\end{itemize}

Por lo tanto, el algoritmo tiene un orden temporal de O($n^2$), cumpliendo con lo pedido en el enunciado.

\subsection{Ejemplos y Soluciones}

Aca dibujito de grafo

\subsection{Experimentacion}

Experimentar y tirar graficos


\chapter{Ap\'endice}
%\subsection{Medición de los tiempos}

Para este tp como trabajamos bajo el lenguaje de programación C++, decidimos calcular los tiempos utilizando 'chrono' de la librería standard de c++ (chrono.h) que nos permite calcular el tiempo al principio del algoritmo y al final, y devolver la resta en la unidad de tiempo que deseamos. Todos los tiempos estan medidos en nanosegundos.\\ \\


\section{C\'odigo Fuente}

\subsection{Backtrack}
\lstinputlisting[language=C++]{AlgoritmosFinales/Ej2/ej2.cpp}

\subsection{Heuristica Golosa}
\lstinputlisting[language=C++]{AlgoritmosFinales/Ej3/ej3.cpp}

\subsection{Heuristica de Busqueda Local 1}
\lstinputlisting[language=C++]{AlgoritmosFinales/Ej4/ej4.1.cpp}

\subsection{Heuristica de Busqueda Local 2}
\lstinputlisting[language=C++]{AlgoritmosFinales/Ej4/busquedalocal2.cpp}

\subsection{Heuristica de Busqueda Local 3}
\lstinputlisting[language=C++]{AlgoritmosFinales/Ej4/busquedalocal3.cpp}

\subsection{GRASP 1}
\lstinputlisting[language=C++]{AlgoritmosFinales/Ej5/ej5A.cpp}


%\pagebreak
%\clearpage
%\phantomsection %esto lo pide el paquete hyperref
%\addcontentsline{toc}{chapter}{Bibliograf\'ia}

%bibliografia con bibtex
 %\bibliographystyle{plain}
 %\bibliography{biblioAlgo3}

%bibliografia con el entorno thebibliography
%\input{./bibliografia.tex}

\pagebreak



\end{document}

