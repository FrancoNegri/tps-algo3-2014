\section{Introducci\'on}
En este problema, se nos pide que ideemos un algoritmo que dados $n$ productos quimicos que deben transportarse en camiones de un lugar a otro, tales que si un elemento $i$ va en el mismo cambion que otro elemento $j$, esto conlleva una "peligrosidad" asociada $h_i_j$. El objetivo aqu\'i es utilizar la menor cantidad de cami\'ones posibles, pero que cada cami\'on tenga una peligrosidad menor a una cota $m$.
\\
La entrada del problema consiste en:
\\
\begin{itemize}
\item Un entero \textbf{n} $\rightarrow$ Representar\'an el n\'umero de productos quimicos a transportar.
\item Un entero \textbf{m} $\rightarrow$ Representar\'a la cota de peligrosidad que ningun cami\'on puede superar.
\item \textbf{n-1} filas donde, para cada fila $i$ consta de $n-i$ enteros:
\begin{itemize}
\item $h_i_{i+1}, h_i_{i+2}$ ... $h_i_{n}$ $\rightarrow$ Representar\'an la peligrosidad asociada del elemento $i$ con los elementos $i+1$, $i+2$ ... $n$.
\end{itemize}
\end{itemize}
\\
La salida, por su parte, constar\'a de una fila con:
\\
\begin{itemize}
\item Un entero \textbf{C} $\rightarrow$ Representar\'a la cantidad indispensable de cami\'ones que es necesaria para transportar los productos bajo las condiciones del problema.
\item $n$ enteros $\rightarrow$ Representar\'an en que cami\'on viaja cada producto.
\end{itemize}

\subsection{Ejemplo de entrada valida}
Hagamos un pequeño ejemplo para que pueda ilustrarse bien el problema.
\\
Supongamos que tenemos $3$ productos quimicos, el producto $1$ es muy inestable, por lo que si es transportado con el producto $2$ la peligrosidad asciende a $40$, y si se transporta con el producto $3$ la peligrosidad es de $35$. El producto $2$ en cambio es de naturaleza mas estable, por lo que si es transportado con el producto $3$ solo produce una peligrosidad de $3$.
\\
Por otro lado queremos que la peligrosidad por cami\'on no supere el valor de $39$.
\\
Entonces la entrada para este problema ser\'a:
\\
\\
$\textbf{3 39}$
\\
$\textbf{40 35}$
\\
$\textbf{3}$
\\
\\
Para una entrada de estas dimenciones es posible buscar la mejor solucio\'on a mano.
\\
Las posibles combinaciones son que los tres productos viajen juntos, que los tres viajen separados, que $1$ y $2$ viajen juntos, que $1$ y $3$ viajen juntos y que $2$ y $3$ viajen juntos y el producto sobrante viaje en otro cami\'on.
\\
La primera d\'a una peligrosidad de $40+35+3$ por lo que es inviable, la segunda es valida, pero se necesitan $3$ camiones. Que $1$ y $2$ viajen juntos, tampoco es valida (peligrosidad muy alta), y finalmente las ultimas dos son validas (peligrosidad $35$ y $3$, respectivamente) y solo son necesarios dos camiones.
\\
Luego las dos salidas que podr\'a devolver el algoritmo son:
\begin{itemize}
\item $2$ $1$ $2$ $1$
\end{itemize}
o
\begin{itemize}
\item $2$ $1$ $2$ $2$
\end{itemize}

\newpage
\section{Idea General de Resoluci\'on}
De el ejemplo anterior, se desprende algo crucial para la resoluci\'on de este algoritmo, y es que no es necesario probar todas las combinaciones de camiones, por ejemplo, no probamos la combionacion $\textbf{2 2 2}$ o $\textbf{3 3 3}$, estas combinaciones son identicas a la soluci\'on $\textbf{1 1 1}$ y no aportan nada nuevo al problema.
\\
De la misma manera obviamos la posible soluci\'on $\textbf{1 3 3}$ ya que estaba contemplada con la solucio\'on $\textbf{1 2 2}$.
\\
Esto ya nos d\'a un excelente punto de arranque para plantear el algoritmo:
\\
Por una parte, sabemos que al menos existe una solucion, y es aquella donde todos los productos viajan separados. Por otro lado sabemos que si 