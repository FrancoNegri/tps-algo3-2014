\subsection{Introduccion} 
Se esta diseñando un software de arquitectura, para el cual es necesario que dado un conjunto de edificios representados como rectangulos apoyados sobre una base en comun, se devuelva el perfil definido en el horizonte.\\
Estos edificios vienen representados por tuplas de tres elementos que representan donde comienza el edificio, su altura y donde termina, de las cuales tenemos que ir tomando en cada momento donde comienza un edificio la altura maxima alcanzada en ese punto.

 \subsection{Ejemplos y Soluciones}
 Consideremos el siguiente ejemplo del problema:\\
 $ [<3,2,5>;<1,4,2>;<4,1,6>;<6,8,10>]$ \\
  Cada una de estas tuplas de tres elementos se indica donde comienza el edificio en la primera coordenada, su altura en la segunda y donde termina en la tercera coordenada.\\
  
  Lo primero que hacemos es ordenar estas tuplas en orden creciente por lo que representa donde comienza el edificio (llamemosla pared izquierda), quedandonos de la siguiente manera:\\ 
$[<1,4,2>;<3,2,5>;<4,1,6>;<6,8,10>$] \\

Por otro lado ordenamos los edificios por la coordenada donde terminan (llamemosla pared derecha).\\
$[<1,4,2>;<3,2,5>;<4,1,6>;<6,8,10>$] \\

\subsection{Desarrollo}
Como mencionamos anteriormente, tenemos como datos de entrada la posici\'on donde comienza y termina cada edificio y adem\'as su altura, con lo cual representaremos a los edificios con tuplas de 3 elementos (posici\'on de inicio o pared izquierda, altura, y posici\'on donde termina o pared derecha). Primero organizaremos a los edificios en dos arreglos, donde cada arreglo contendr\'a el total de edificios, uno con un orden ascendente seg\'un pared izquierda y el otro tambi\'en con un orden ascendente pero seg\'un pared derecha. Adem\'as tendremos un conjunto en el que iremos agregando los edificios que ``comience'' y quitando los edificios que ``terminen'', es decir, aqu\'i� estar\'an los edificios ``activos'' (que ``empezaron'' y no ``terminaron'') en cada momento.
La idea del algoritmo es ir recorriendo las posiciones (del eje x) en las que haya una o m\'as paredes. En cada punto lo que haremos es agregar a mi conjunto de activos los edificios que en ese punto tengan su pared izquierda, o sea que est\'an "comenzando", y quitar a los edificios que all\'i tengan su pared derecha, o sea que est\'an ``terminando''. Una vez completada esta labor, buscaremos el edificio activo que tenga la altura m\'axima y nos lo guardaremos, llam\'emoslo Max. Se puede ver claramente que el borde superior de la silueta en un punto dado va a estar determinado por el edificio m\'as alto que haya en ese punto, es decir, el edificio activo m\'as alto. Como en cada paso podemos conseguir el edificio activo m\'as alto, proseguiremos as\'i� hasta el \'ultimo punto y habremos arm\'andonos la silueta.

\subsection{Demostraci\'on}
Demostraremos por inducci�n que en cada paso de este algoritmo obtenemos la altura del edificio m�s alto en ese punto.

P(i) = "Nuestro conjunto de edificios activos contiene todos los edificios que empezaron entre el punto de inicio y el punto i inclusive y que a�n no han terminado, es decir, terminan en un punto estrictamente mayor que i"

Caso base, P(1):
En este caso nos encontramos con nuestro primer conjunto de paredes, que puede tener una o m�s paredes. Este s�lo puede tener paredes izquierda ya que es donde comienzan nuestros primeros edificios y ning�n edificio ha empezado antes como para que aparezca una pared derecha indicando su finalizaci�n. Por lo tanto s�lo tenemos un conjunto de edificios que comienzan en este punto, los cuales agregaremos a nuestro conjunto de edificios activos. Es claro ver que el conjunto de edificios activos ahora tiene todos los edificios que comenzaron y no han terminado hasta el primer punto inclusive ya que ning�n edificio termina aqu� como previamente hemos dicho y entonces todos estos empezaron y no terminaron.

Paso inductivo, P(n) $\rightarrow$ P(n+1):
Nuestra hip�tesis inductiva nos dice que nuestro conjunto de edificios activos contiene todos los edificios que empezaron entre el punto de inicio y el punto n inclusive, y que a�n no han terminado. Ahora veamos qu� ocurre en n+1. En este punto podemos encontrar un conjunto que contiene tanto paredes izquierda como derecha, con lo cual separaremos a este conjunto de paredes en dos subconjuntos. Por un lado el conjunto de paredes izquierda, llam�mosle I, y por otro el de paredes derecha, llam�mosle D. Recorreremos primero el conjunto I agregando cada edificio de este conjunto al conjunto de edificios activos, es decir, agregaremos todos los edificios que comienzan en este punto.
Acto seguido, recorreremos el conjunto D quitando cada edificio que aparezca en este conjunto de nuestro conjunto de edificios activos ya que todos estos edificios est�n terminando y por lo tanto ya no deben pertenecer a nuestro conjunto. Veamos ahora que efectivamente se est� cumpliendo P(n+1). Los edificios agregados en este paso no pueden terminar aqu� mismo, ya que eso implicar�a que est�n empezando y terminando en la misma posici�n, lo cual no es v�lido. Por lo tanto estar�an terminando en un punto estrictamente mayor a n+1. 
Los edificios que terminaban en el paso n+1 fueron ya removidos del conjunto. Ahora miremos qu� pasa con los dem�s edificios que ya estaban en nuestro conjunto de edificios activos y no fueron eliminados. Por hip�tesis inductiva, estos edificios no tienen su pared derecha entre el punto de inicio y el punto n, es decir, terminan en un punto estrictamente mayor que n. Pero como no fueron borrados de nuestro conjunto, quiere decir que no tienen su pared derecha en n+1, o sea que terminan en un punto estrictamente mayor a n+1. Entonces vale P(n+1).


Teniendo nuestro conjunto de edificios activos, en cada punto podemos buscar el edificio de mayor altura y registrar los cambios de altura cada vez que empieza y/o termina un edificio. Luego, es trivial armarnos la silueta ya que, con las fluctuaciones de las alturas m�ximas a lo largo de nuestra ciudad, ya tenemos su borde superior en cada tramo.


\subsection{Complejidad}

definimos edificio = < izq :natural x alto : natural x der : natural >
definimos edificioenCero al que tiene todos sus elementos en cero.
\begin{algorithm}
\begin{algorithmic}[1]\parskip=1mm
 \caption{LaSilueta( ciudad: arreglo(edificios) , cantidadEdificios : natural)}
	
	\STATE{arregloXIzq \leftarrow ordenarXIzquierda(ciudad)}\\
	\STATE{arregloXDer \leftarrow ordenarXDerecha(ciudad)}\\
	\STATE{EdificiosActivos \leftarrow Multiconjunto(edificio)}\\
	\\
	\STATE{posIzquierdo , posDerecho \leftarrow 0}\\
	\STATE{max \leftarrow edificioenCero \\
	\STATE{mientras posIzquierdo < cantidadEdificios \&\& posDerecho < cantidadEdificios }\\
	\STATE{ \quad SI arregloXIzq.indice(posIquierdo).izq \leq arregloXDer.indice(posDerecho).der}\\
			\STATE{\quad\quad auxiliar \leftarrow arregloXIzq.indice(posIzquierdo)}\\
			\STATE{\quad\quad mientras auxiliar.izq = arregloXIzq.indice(posIquierdo).izq \&\& posIzquierdo != cantidadEdificios}\\
			\STATE{\quad\quad\quad agregar(arregloXIzq.indice(posIquierdo), EdificiosActivos )}\\
			\STATE{\quad\quad\quad SI arregloXIzq.indice(posIquierdo).alto > max.alto}\\
			\STATE{\quad\quad\quad\quad max \leftarrow arregloXIzq.indice(posIquierdo) }\\
		\STATE{\quad SI NO }\\
			\STATE{\quad\quad auxiliar \leftarrow arregloXDer.indice(posDerecha)}\\
			\STATE{\quad\quad mientras auxiliar.der = arregloXDer.indice(posDerecha).der && posDerecha != cantidadEdificios}\\
			\STATE{\quad\quad\quad SI  arregloXDer.indice(posDerecha).id = max.id}\\
			\STATE{\quad\quad\quad\quad dameMaximoSiguiente( arregloXDer.indice(posDerecha), EdificiosActivos)}\\
			\STATE{{\quad\quad\quad\quad sacar( arregloXDer.indice(posDerecha), EdificiosActivos)}
\\
\end{algorithmic}
Los algoritmos ordenarXizquierda y ordenarXDerecha es el conocido algoritmo mergeSort sacado del libro de brassard, la complejidad del mismo es \Ode{N*Log(N)} siendo N la cantidad de elementos en el arreglo, �sea todos los edificios, estos algoritmos ordenan los arreglos de tuplas, uno por la coordenada izq y el otro por la coordenada derecha respectivamente.\\
El multiconjunto EdificiosActivos esta representado con la estructura Multiset de la librer�a STL de c++, la relaci\'on de orden sobre los elementos que se defini\'o es por la coordenada alt y en el caso que la altura sea igual por la componente der.\\
Para las operaciones para agregar, sacar elementos la complejidad es \Ode{log(n)} siendo n la cantidad de elementos en el multiconjunto y para obtener el proximo maximo lo que se realiza es obtener la referencia al elemento que 
representa el m\'aximo, esto lleva \Ode{Log(n)} y luego se obtiene el posterior, en caso de no ser posible obtenemos el anterior, en el peor caso se tienen todos los edificios y la complejidad seria \Ode{log(N)}.\\
El ciclo de las lineas 9-12 se realiza para cada punto x donde haya paredes izquierdas, con lo cual la complejidad total sera la suma de todas las paredes izquierda, siendo esta N la cantidad de edificios. De la misma manera el ciclo de las lineas 15 a 18 se ejecuta para todas las paredes derechas de los edificios con lo cual el total de las iteraciones sera la suma de las paredes derechas �sea N. Como el ciclo principal se ejecuta mientras la cantidad de paredes recorridas izquierdas sea menor a la cantidad de edificios y la cantidad de paredes derechas sea menor a la cantidad de edificios y las iteraciones internas modifican las paredes izquierdas y derechas recorridas, el total de las iteraciones es 2*N, siendo N la cantidad total de edificios, con lo cual como la complejidad de agregar , sacar y obtener el m\'aximo es \Ode{log(N)} en el peor caso, el total es \Ode{2*N*Log(N)} $\subset$ \Ode{N*Log(N)} siendo esta la complejidad total del algoritmo y esta estrictamente menor a \Ode{N^2} como se solicitaba en el enunciado.

\end{algorithm}	
\newpage
\subsection{Experimentaci\'on}
Se realizaron experimentaciones sobre varios tipos de caso para observar el comportamiento del algoritmo, en todos los casos se generaron 1000 entradas, a las gr�ficas resultantes se las comparo con la funci�n n*log(n)*1000, siendo n la cantidad de edificios de entrada.\\
Para la primer ejecuci�n se realizo un an\'alisis sobre el algoritmo en edificios donde la distancia entre sus paredes izquierda y derecha era corta, estas se encontraban entre las posiciones 1 y 30, pero sus alturas variaban en el orden entre 1 y 10000, dando el siguiente gr�fico como resultado.\\

\includegraphics[scale=0.7]{Ej2/VariacionAlturas.jpg}\\

El segundo gr\'afico corresponde a instancias generadas de manera random donde la altura se encontraba entre 1 y 40, pero las distancias entre la primer y la segunda pared de cada edificio varia entre los 1 y 2000 , elegido de manera random la posici�n de ambos lados pero siempre que sea entrada valida, \'osea el valor de la pared izquierda es mayor estricta a la de la pared derecha.

\includegraphics[scale=0.7]{Ej2/VariacionDistancias.jpg}


Luego para el tercer gr\'afico consideramos que era importante ver como se comportaba el multiconjunto cuando se agregaban todos los edificios de la entrada, de esta manera se generaron entradas donde los edificios eran todos iguales, as\'i se ingresaban todos al multiconjunto y ve\'iamos como se comportaba el algoritmo, obteniendo este gr\'afico como resultado donde se ve que su complejidad es del orden logar\'itmico en la cantidad de edificios de entrada por la cantidad edificios.

\includegraphics[scale=0.7]{Ej2/iguales.jpg}\\
Por ultimo realizamos un gr�fico con entradas random donde las mismas variaban combinando todas las anteriores, obteniendo de nuevo que el orden era logar�tmico al compararlo con la funci�n.

\includegraphics[scale=0.7]{Ej2/Random.jpg}\\

 Concluimos por lo tanto que el algoritmo respeta los ordenes de complejidad requeridos.



