\subsection{Introduccion} 
Se esta diseñando un software de arquitectura, para el cual es necesario que dado un conjunto de edificios representados como rectangulos apoyados sobre una base en comun, se devuelva el perfil definido en el horizonte.\\
Estos edificios vienen representados por tuplas de tres elementos que representan donde comienza el edificio, su altura y donde termina, de las cuales tenemos que ir tomando en cada momento donde comienza un edificio la altura maxima alcanzada en ese punto.

 \subsection{Ejemplos y Soluciones}
 Consideremos el siguiente ejemplo del problema:\\
 $ [<3,2,5>;<1,4,2>;<4,1,6>;<6,8,10>]$ \\
  Cada una de estas tuplas de tres elementos se indica donde comienza el edificio en la primera coordenada, su altura en la segunda y donde termina en la tercera coordenada.\\
  
  Lo primero que hacemos es ordenar estas tuplas en orden creciente por lo que representa donde comienza el edificio (llamemosla pared izquierda), quedandosnos de la siguiente manera:\\ 
$[<1,4,2>;<3,2,5>;<4,1,6>;<6,8,10>$] \\

Por otro lado ordenamos los edificios por la coordenada donde terminan (llamemosla pared derecha).\\
$[<1,4,2>;<3,2,5>;<4,1,6>;<6,8,10>$] \\

\subsection{Demostracion}


\subsection{Complejidad}

\subsection{Experimentacion}

  



