\subsection{Introduccion} 
Se esta diseñando un software de arquitectura, para el cual es necesario que dado un conjunto de edificios representados como rectangulos apoyados sobre una base en comun, se devuelva el perfil definido en el horizonte.\\
Estos edificios vienen representados por tuplas de tres elementos que representan donde comienza el edificio, su altura y donde termina, de las cuales tenemos que ir tomando en cada momento donde comienza un edificio la altura maxima alcanzada en ese punto.

 \subsection{Ejemplos y Soluciones}
 Consideremos el siguiente ejemplo del problema:\\
 $ [<3,2,5>;<1,4,2>;<4,1,6>;<6,8,10>]$ \\
  Cada una de estas tuplas de tres elementos se indica donde comienza el edificio en la primera coordenada, su altura en la segunda y donde termina en la tercera coordenada.\\
  
  Lo primero que hacemos es ordenar estas tuplas en orden creciente por lo que representa donde comienza el edificio (llamemosla pared izquierda), quedandonos de la siguiente manera:\\ 
$[<1,4,2>;<3,2,5>;<4,1,6>;<6,8,10>$] \\

Por otro lado ordenamos los edificios por la coordenada donde terminan (llamemosla pared derecha).\\
$[<1,4,2>;<3,2,5>;<4,1,6>;<6,8,10>$] \\

\subsection{Demostracion}
Demostraremos por inducción que en cada paso de mi algoritmo obtengo la altura del edificio más alto en ese punto.

P(i) = "Mi edificio Max es el edificio más alto en el punto i"

Caso base, P(1):
En este caso nos encontramos con nuestro primer conjunto de paredes, que puede tener una o más paredes. Este sólo puede tener paredes izquierda ya que es donde comienzan nuestros primeros edificios y ningún edificio ha empezado antes como para que aparezca una pared derecha indicando su "finalización". Por lo tanto tenemos un conjunto de edificios que "comienzan" en este punto y nos basta recorrer ese conjunto para encontrar el edificio de mayor altura, al cual llamaremos Max. Entonces, sin duda, Max es el edificio más alto en este punto.

Paso inductivo, P(n) $\rightarrow$ P(n+1):
Nuestra hipótesis inductiva nos dice que el edificio más alto en el punto n es Max. Ahora veamos qué ocurre en n+1. En este punto podemos encontrar un conjunto que contiene tanto paredes izquierda como derecha, con lo cual separaremos a este conjunto de paredes en dos subconjuntos. Por un lado el conjunto de paredes izquierda, y por otro el de paredes derecha. Recorreremos primero el conjunto de paredes izquierda fijándonos si algún edificio de este conjunto supera en altura a Max. Cada vez que alguno supere a nuestro Max actual, ese será el nuevo Max. Si después terminar el recorrido mi Max anterior cambió, entonces ese nuevo Max será el edificio más alto en este punto (y no es posible que este nuevo Max tenga su pared derecha en este punto ya que eso implicaría que ese edificio está "empezando" y "terminando" en el mismo punto, lo cual no es válido). Y caso contrario sigo manteniendo a mi Max de antes y debo fijarme las paredes derecha para ver si "termina".
Ahora recorreremos el conjunto de paredes derecha fijándonos si está "terminando" el edificio Max en cada paso. Entonces cada vez que aparezca una pared derecha del edificio Max actual, lo reemplazo por el edificio activo, o sea que "empezó" y no "terminó", más alto. Para este reemplazo, lógicamente también se tienen en cuenta los edificios que "comenzaron" en este punto. Por lo tanto, después de terminar el recorrido mi edificio Max es el más alto de los edificios en ese punto.

Teniendo las alturas máximas para cada punto, es trivial armarme la silueta ya que con las alturas máximas ya tengo su borde superior en cada punto.


\subsection{Complejidad}

\subsection{Experimentacion}

  



