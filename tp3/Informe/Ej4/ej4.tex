\section{Introducci\'on}

En esta sección se implementarán heuristicas de busqueda local para tratar de resolver el problema de $k-PMP$, intentando con diferentes metodos para alcanzar una solución y diferentes vecindades.

Para ello, partiendo de una solución generada al azar, se intentará a travez de sucesivas iteraciones analizar sierta vecindad para intentar mejorar la solución existente y aproximarse mas a una 'buena' solución en un tiempo aceptalble. 

Por lo tanto debemos tener en cuenta de no hacer las vecindades ni muy grandes, ya que esto puede llevar a una perdida de performance, ni muy pequeñas, ya que esto puede llevar a que el algoritmo explore solo una pequeña cantidad de soluciones y devuelva una solución muy alejada del optimo.

Como primera idea para una vecindad tomaremos cada nodo del grafo, vemos cuanto peso agrega en la suma intraconjunto en que se encuentra, lo quitaremos de este conjunto e intentamos meterlo en todos los demas, viendo si en alguno logra minimizar esta suma. En caso afirmativo, lo sacamos de su antiguo conjunto y lo ponemos en el nuevo. Realizamos esto hasta que deja de ser posible mejorar la solución y en este punto la devolvemos.

Esta heuristica, como se verá en el apartado de testing, devuelve resultados vastante aproximados, al compararlo con la solución exacta y con las otras heuristicas.

Otra vecindad que plantearemos será buscar el nodo que mas peso esta generando en la suma intrapartición, quitarlo de la partición donde se encuentra y agregarlo a alguna otra. 