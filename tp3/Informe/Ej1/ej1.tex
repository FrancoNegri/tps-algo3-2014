Para este trabajo práctico se nos pide, a partir de un grafo simple $G=(V,E)$ con pesos en las aristas, encontrar la k-particion tal que minimice el peso de las aristas intrapartición.

Para ello primero intentaremos relacionar este problema con otros problemas conocidos, pensaremos varias maneras de abordarlo y emprenderemos la busqueda de algoritmos eficientes para resolverlo.

Como veremos luego, este problema es 'dificil' de resolver, por lo que para instancias grandes, el algoritmo dejará de ser viable, por lo que tambien desarrollaremos distintas heruristicas que resuelvan el problema de manera aproximada. Probaremos con una heuristica golosa, dos heruristicas de busquedas locales y a partir de estos, un GRASP.

\subsection{Entrada y salida}

Todos los algoritmos tomarán como entrada, lo siguiente:

\begin{itemize}
	\item Un entero $\textbf{n}$ $\rightarrow$ Representará el numero de nodos del grafo $G$.

	\item Un entero $\textbf{m}$ $\rightarrow$ Representará la cantidad de aristas del grafo.

	\item Un entero $\textbf{m}$ $\rightarrow$ Representará la cantidad de conjuntos distintos que disponemos para poner los ejes.

	\item $\textbf{m}$ filas donde, para cada fila $i$ consta de $3$ enteros:
	\begin{itemize}
		\item $\textbf{u v w}$ $ \rightarrow $ donde $\textbf{u}$ y $\textbf{v}$ son los nodos adyacentes y $\textbf{w}$ el peso de las aristas entre ellos.
	\end{itemize}
\end{itemize}

La salida, por su parte, constar\'a de una fila con:

\begin{itemize}

	\item $n$ enteros $i_1$ $i_2$ $...$ $i_n$

\end{itemize}

Donde cada $i_k$ representa en que conjunto se encuentra el nodo $k$

\subsection{Ejemplo}

Para ejemplificar el problema a resolver, pensemos en un grafo $G$ con $4$ nodos, $5$ vertices y $2$ particiones.

Supongamos ademas que los nodos estan conectados de la siguiente manera:

\begin{itemize}

	\item $1-2$ con peso $2$
	\item $1-3$ con peso $3$
	\item $1-4$ con peso $3$
	\item $2-4$ con peso $1$
	\item $3-4$ con peso $2$

\end{itemize}

A primera vista podríamos elegir el nodo con mas aristas, el nodo $1$, y ponerlo en una partición obteniendo k-PMP = \{\{($1$)\},\{\}\}. Por otro lado tenderíamos a poner nodos que no esten conectados en otra partición, por ejemplo los nodo $2$ y $3$ ya que estos no los conecta ninguna arista, obteniendo k-PMP = \{\{($1$)\},\{($2$),($3$)\}\}. Ahora en nuestro ejemplo nos queda el nodo $4$ y al tener $k = 2$ lo tenemos que poner en alguna, elegiremos la partición donde minimice el peso total, en este caso es indiferente, dado que agregar el nodo $4$ a la primer partición suma $3$ y a la segunda partición también suma $3$, resultando k-PMP = \{\{($1$),($4$)\},\{($2$),($3$)\}\}. 

De esta forma, mas adelante vamos a ver que un algoritmo goloso que intentase resolver este problema con un esquema similar a lo descripto podria llegar a generar soluciónes tan malas como se quiera.

\subsection{Relación con el problema 3 del Trabajo Práctico 1}

... completar ...

\subsection{Relación con el problema de Colorear un Grafo}

Vamos a relacionar el problema de $k-PMP$ con el problema de coloreo. Supongamos que podemos encontrar un k-coloreo para los vértices del grafo G, entonces podríamos subdividir al conjunto de vértices V en k subgrupos según su color, es decir, en un mismo grupo sólo habrá vertices que compartan color, generando la siguiente partición: k-PMP = \{ $V_1$, ..., $V_k$ \}. 

Por la definición de coloreo, dos vértices de un mismo color no pueden tener aristas entre sí, por lo tanto los grupos que armamos son conjuntos independientes. Esto quiere decir que cada k-partición no tiene aristas intrapartición ya que cada una es un conjunto independiente, luego el peso total de la misma es $0$. Además es un peso mínimo ya que no hay aristas con peso negativo, obteniendo la mejor solución a nuestro problema.


En primer instancia podemos relacionar el problema de $k-PMP$ con el problema de coloreo. Supongamos que podemos encontrar un k-coloreo para los vértices del grafo G, entonces podríamos subdividir al conjunto de vértices V en $k$ subgrupos según su color, es decir, en un mismo grupo sólo habrá vertices que compartan color. Por la definición de coloreo, dos vértices de un mismo color no pueden tener arista entre sí, por lo tanto los grupos que armamos son conjuntos independientes. Esto quiere decir que mi $k$-partición no tendría aristas intrapartición ya que cada partición sería un conjunto independiente, y entonces el peso total de la misma sería $0$. Y además sería un peso mínimo ya que no hay aristas con peso negativo y entonces tendríamos la solución a nuestro problema. Hasta aquí hemos visto que con un coloreo igual a $k$ se obtiene la solución al problema, pero observemos que los conjuntos de la $k$-partición resultado no tienen que ser necesariamente no vacíos, lo que nos lleva a pensar que también nos bastaría conseguir un coloreo menor a $k$ para resolver nuestro problema y ahora veremos cómo puede ser esto posible. Dado un $k'$-coloreo con $k'<k$, por lo dicho anteriormente podemos armarnos $k'$ subgrupos de vértices agrupandolos por color, los cuales serán conjuntos independientes, es decir, no existirán aristas que incidan en dos nodos de un mismo grupo. Sin embargo, ya no me queda ningún vértice para meter en algún grupo y el problema me pide $k$ particiones, por lo que me estarían faltando otras $k-k'$ particiones más que agregar a mi $k$-partición. Pero si recordamos nuestra observación que decía que las particiones no deben ser necesariamente no vacías, entonces podríamos agregar a nuestra $k$-partición $k-k'$ particiones de vértices vacías, con lo cual no estaríamos agregando ninguna arista intrapartición. Entonces mis $k'$ particiones iniciales, por lo visto previamente, no tienen ninguna arista intrapartición y los conjuntos vacíos que agregué posteriormente claramente tampoco tienen aristas intrapartición, por lo que he llegado nuevamente a una $k$-partición de peso $0$, la cual es solución de mi problema y además es óptima.

Por otro lado, dada una solución al problema de k-PMP de peso estrictamente mayor a $0$ para un grafo G determinado, podemos afirmar que no existe un $k$-coloreo para ese grafo. Si existiera un $k$-coloreo para dicho grafo, entonces, por lo probado en el anterior párrafo, también existiría una k-PMP de peso $0$ para tal grafo, lo cual es absurdo ya que partimos de una k-PMP de peso estrictamente mayor a $0$.

También es posible relacionar nuestro problema con el problema 3 del TP 1. Si analizamos la situación con cierto detenimiento, podemos ver que el grafo G podría modelar perfectamente los productos químicos a transportar y sus respectivos coeficientes de peligrosidad de a pares, es decir, cada nodo representaría un producto y cada arista indicaría que existe un coeficiente de peligrosidad entre los productos o nodos sobre los que incide y su peso sería el valor de este coeficiente. Por otra parte, el peso de una partición se mide como la suma de las aristas intrapartición, lo cual es análogo al nivel de peligrosidad de un camión, que es la suma de las peligrosidades de a pares de los items transportados, con lo cual concluímos en que las particiones son un buen modelo de los camiones. La diferencia con el problema de los camiones, es que este último lo que busca minimizar es la cantidad de camiones utilizados para el transporte de los productos, mientras que en el problema k-PMP ya viene dado un número $k$ de particiones y lo que se quiere minimizar es la suma de los pesos de todas las particiones, o sea, el equivalente a la suma de los niveles de peligrosidad de cada camión. Además, el nivel de peligrosidad de cada camión, para cumplir con las normas vigentes de seguridad, no puede superar un cierto umbral M y en cambio las particiones no tienen un límite de peso. Utilizando la k-PMP estaría resolviendo un problema similar al de los camiones, es decir, estaría buscando minimizar la suma de los niveles de peligrosidad de los camiones y además tendría una cantidad limitada $k$ de camiones para usar y además no existiría una cota de peligrosidad por camión.