\section{Idea}


Para la metaheurística de GRASP, tomaremos el algoritmo goloso previamente implementado y lo combinaremos con las diferentes búsquedas locales también previamente implementadas.

La idea es la siguiente, para cada paso del algoritmo, corremos una versión modificada del goloso de manera tal de permitir cierta aleatoriedad en los resultados. El algoritmo goloso será modificado de la siguiente manera:

\begin{algorithm}
  \begin{algorithmic}[1]\parskip=1mm
 \caption{ Goloso()}
 		\STATE{Numero los vértices de $1$ a $n$} 
		\STATE{Creo una cantidad $k$ de conjuntos donde iré guardando vértices}
 		\STATE{Para cada nodo $i$ de $1$ a $n$: }
		\STATE{\quad Para cada conjunto}
			\STATE{\quad\quad Sumo todos los pesos de las aristas de ($i$,$j$) con $j$ los vértices que están en el conjunto}
 		\STATE{\quad De los mejores $x$ resultados, tomo uno al azar y pongo a $i$ en ese conjunto}
		\STATE{Devuelvo la respuesta}
\end{algorithmic}
\end{algorithm} 

Notar que si $x=1$ entonces estamos obteniendo el mismo algoritmo goloso que en el apartado anterior. En cambio, si tomo $x = k$ (esto es, tomo los $k$ mejores resultados, o sea todos), estaría generando una solución completamente aleatoria. Cualquier solución en medio tomará una solución golosa, pero con cierto grado de aleatoriedad.

Luego a la solución obtenida por el algoritmo goloso modificado se le aplicarán una de las búsquedas locales implementadas en un intento de acercar aún mas la respuesta al óptimo.

Como criterio de corte se correrá el goloso aleatorizado y la busqueda local hasta que luego de que un número $x$, a determinar, de intentos no haya sido posible mejorar la solución. En este punto se entrega la mejor respuesta obtenida hasta el momento.

Dado que tenemos tres busquedas locales, se decide desarrollar tres GRASPs diferentes. Uno con cada una de las busquedas locales. A continuación se formaliza de manera más precisa los algoritmos de cada una:

\begin{algorithm}
  	\begin{algorithmic}[1]\parskip=1mm
		 \caption{ GRASP 1(SoluciónInicial) }
		 \STATE{while(true)}
		 	\STATE{\quad Corro el Algoritmo Goloso modificado}
		 	\STATE{\quad Utilizo búsqueda local 1 para mejorar la solución obtenida previamente}
		 	\STATE{\quad Si conseguí una mejor solución que antes, la guardo}
		 	\STATE{\quad Si tras $z$ iteraciones no se pudo conseguir una mejor solución}
		 	\STATE{\quad\quad Devuelvo la solución}
	\end{algorithmic}
\end{algorithm}

\begin{algorithm}
  	\begin{algorithmic}[1]\parskip=1mm
		 \caption{ GRASP 2(SoluciónInicial) }
		 \STATE{while(true)}
		 	\STATE{\quad Corro el Algoritmo Goloso modificado}
		 	\STATE{\quad Utilizo búsqueda local 2 para mejorar la solución obtenida previamente}
		 	\STATE{\quad Si conseguí una mejor solución que antes, la guardo}
		 	\STATE{\quad Si tras $z$ iteraciones no se pudo conseguir una mejor solución}
		 	\STATE{\quad\quad Devuelvo la solución}
	\end{algorithmic}
\end{algorithm}

\begin{algorithm}
  	\begin{algorithmic}[1]\parskip=1mm
		 \caption{ GRASP 3(SoluciónInicial) }
		 \STATE{while(true)}
		 	\STATE{\quad Corro el Algoritmo Goloso modificado}
		 	\STATE{\quad Utilizo búsqueda local 3 para mejorar la solución obtenida previamente}
		 	\STATE{\quad Si conseguí una mejor solución que antes, la guardo}
		 	\STATE{\quad Si tras $z$ iteraciones no se pudo conseguir una mejor solución}
		 	\STATE{\quad\quad Devuelvo la solución}
	\end{algorithmic}
\end{algorithm}

En estos tres casos $z$ será un numero entero entre $1$ e infinito.

En el siguiente apartado de experimentación se intentarán calibrar tanto el parámetro $x$ como el parámetro $z$ para así obtener los mejores resultados posibles con estas metaheurísticas, tratando siempre de equilibrar performance y exactitud de los resultados.

\section{Experimentación}

Para cada uno de los algoritmos seteamos ambos parametros $x$ y $z$ de manera arbitraria para obtener resultados más o menos adecuados.

Luego para aproximar más estos resultados comenzamos a variar $x$ dejando fijo $z$ y viendo cuales son los cambios en la performance de cada uno de los GRASPs.

Para eso tomamos un grafo completo de $500$ nodos tal que las aristas tienen pesos enteros aleatorios entre $1$ y $100$. Corremos el algoritmo con $x=1$, lo que equivale a una solución inicial obtenida por el algoritmo goloso determinístico, $x=k$ una solución inicial completamente aleatoria, y luego soluciones intermedias para $x = \lfloor0.25 k\rfloor$, $x = \lfloor0.50 k\rfloor$ y $x = \lfloor0.75 k\rfloor$.

Dado que la solución depende fuertemente de $k$ para el grafo antes mencionado, le variaremos el $k$ para tres instancias diferentes con $k = 50$, $k = 200$ y $k = 400$.

Luego los resultados obtenidos para el grafo con $500$ nodos y $k = 50$ son los siguientes:

\includegraphics[scale=0.5]{Ej5/comparacionEntreGrasp50.jpg}

Hasta aquí los resultados parecerían indicar que tomando $x = 1$ pequeño se obtienen resultados muy superiores a los demas casos, tanto en los tiempos obtenidos, como en los conjuntos encontrados.

Se procede a realizar la experimentación para el mismo grafo de $500$ nodos, pero ahora con $k = 200$:

\includegraphics[scale=0.5]{Ej5/comparacionEntreGrasp200.jpg}

Nuevamente podemos observar que para $x = 1$ se obtienen los mejores resultados tanto en performance como en tiempo.

Ahora realizando la experimentación con $k = 400$, se obtiene:

\includegraphics[scale=0.5]{Ej5/comparacionEntreGrasp400.jpg}

De esta manera determinamos que los mejores resultados para este testeo se obtienen con un $x$ fijo igual a $1$. Para ajustar mejor el valor de $x$, ahora tomamos otro grafo de $500$ nodos, y variamos nuevamente el $x$ para valores proximos a $1$. Nuevamente variamos tambien el $k$ para ver como se ven afectados los resultados.



