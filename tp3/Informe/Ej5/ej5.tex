\section{Idea}


Para la metaheurística de GRASP, tomaremos el algoritmo goloso previamente implementado y lo combinaremos con las diferentes búsquedas locales también previamente implementadas.

La idea es la siguiente, para cada paso del algoritmo, corremos una versión modificada del goloso de manera tal de permitir sierta aleatoriedad en los resultados. El algoritmo goloso será modificado de la siguiente manera:

\begin{algorithm}
  \begin{algorithmic}[1]\parskip=1mm
 \caption{ Goloso()}
 		\STATE{Numero los vértices de $1$ a $n$} 
		\STATE{Creo una cantidad $k$ de conjuntos donde iré guardando vértices}
 		\STATE{Para cada nodo $i$ de $1$ a $n$: }
		\STATE{\quad Para cada conjunto}
			\STATE{\quad\quad Sumo todos los pesos de las aristas de ($i$,$j$) con $j$ los vértices que están en el conjunto}
 		\STATE{\quad Del los mejores $x$ resultados, tomo uno al azar y pongo a $i$ en ese conjunto}
		\STATE{Devuelvo la respuesta}
\end{algorithmic}
\end{algorithm} 

Notar que si $x=1$ entonces estamos obteniendo el mismo algoritmo goloso que en el apartado anterior. En cambio, si tomo $x = k$ (esto es, tomo los $k$ mejores resultados, osea todos), estaría generando una solución complemtamente aleatoria. Cualquier solución en medio tomará una solución golosa, pero con sierto grado de aleatoriedad.

Luego a la solución obtenida por el algoritmo goloso modificado se le aplicarán una o varias de las búsquedas locales implementadas en un intento de mejorar aún mas la respuesta al óptimo.

Como primer criterio de corte el algoritmo se correrá un número finito de veces, que será determinado en el momento de experimentación.

Como segundo criterio de corte se realizarán estos mismos pasos hasta que luego de un número $x$ a determinar de intentos no haya sido posible mejorar la solución. En este punto se entrega la mejor respuesta obtenida hasta el momento.

Tras realizar diversas experimentaciones nos quedamos con dos versiones de GRASP, una que en el paso de búsqueda local usa la búsqueda local 2, y otra que en el paso de búsqueda local primero utiliza una búsqueda local 1 y luego una búsqueda local 3.

A continuación se formaliza de manera más precisa el algoritmo:

\begin{algorithm}
  	\begin{algorithmic}[1]\parskip=1mm
		 \caption{ GRASP1(SoluciónInicial) }
		 \STATE{while(true)}
		 	\STATE{\quad Corro el Algoritmo Goloso desarrollado previamente}
		 	\STATE{\quad Utilizo búsqueda local 2 para mejorar la solución obtenida previamente}
		 	\STATE{\quad Si conseguí una mejor solución que antes, la guardo}
		 	\STATE{\quad Si tras $z$ iteraciones no se pudo conseguir una mejor solución}
		 	\STATE{\quad\quad Devuelvo la solución}
	\end{algorithmic}
\end{algorithm}

\begin{algorithm}
  	\begin{algorithmic}[1]\parskip=1mm
		 \caption{ GRASP2(SoluciónInicial) }
		 \STATE{while(true)}
		 	\STATE{\quad Corro el Algoritmo Goloso desarrollado previamente}
		 	\STATE{\quad Utilizo búsqueda local 1 para mejorar la solución obtenida previamente}
		 	\STATE{\quad Utilizo búsqueda local 3 para mejorar la solución}
		 	\STATE{\quad Si conseguí una mejor solución que antes, la guardo}
		 	\STATE{\quad Si tras $z$ iteraciones no se pudo conseguir una mejor solución}
		 	\STATE{\quad\quad Devuelvo la mejor solución}
	\end{algorithmic}
\end{algorithm}

En estos dos casos $z$ será un numero entero entre $1$ e infinito.

En el siguiente apartado de experimentación se intentarán calibrar tanto los parametros $x$ y $z$ para así obtener los mejores resultados posibles, tratando siempre de equilibrar performance y exactitud de los reslutados.

\section{Experimentación}

Para la primera parte de experimentación variaremos el parametro $x$ dejando fijo $z = 10000$ y veremos como varían las soluciones.

Para eso tomamos un grafo completo de $500$ nodos tal que las aristas tienen pesos enteros aleatorios entre $1$ y $100$. Corremos el algoritmo con $x=1$, lo que equivale a una solución inicial obtenida por el algoritmo goloso deterministico, $x=k$ una solución inicial completamente aleatoria, y luego soluciones intermedias para $x = 0.25 k$, $x = 0.50k$ y $x = 0.75 k$.

Dado que la solución depende fuertemente de $k$ al grafo antes mencionado, le variaremos el $k$ para tres instancias diferentes con $k = 50$, $k = 200$ y $k = 400$.

Luego los resultados obtenidos para el grafo con $500$ nodos y $k = 50$ son los siguientes:

\includegraphics[scale=0.5]{Ej5/comparacionEntreGrasp50.jpg}

Para el grafo de $500$ nodos y $k = 200$:

\includegraphics[scale=0.5]{Ej5/comparacionEntreGrasp200.jpg}


Para el grafo de $500$ nodos y $k = 400$:

\includegraphics[scale=0.5]{Ej5/comparacionEntreGrasp400.jpg}

Acá digo: bueno, ya determiné que $x = ??$. Ahora varío la cantidad de iteraciones.